\documentclass{article}
\usepackage{changepage}

\title{IN618 - Assignment 1}
\author{Matthew Hall}
\date{}

\begin{document}
\pagenumbering{gobble}
\maketitle
\newpage
\pagenumbering{arabic}

\tableofcontents
\newpage

\section{Host Discovery}
\paragraph{}
I used the following commands during the host discorvery process:

\texttt{nmap -sn 192.168.100.1}

Identify hosts with IP addresses in the network space of 192.168.100.0/24

\texttt{nmap -O 192.168.100.1}

Attempt to identify the operating system of the host with address 192.168.100.1

\texttt{nmap -O 192.168.100.115}

Attempt to identify the operating system of the host with address 192.168.100.115

\texttt{nmap -O 192.168.100.253}

Attempt to identify the operating system of the host with address 192.168.100.253

\newpage

\section{Information Gathering}
\paragraph{}
By use of the nmap command \texttt{nmap -sV 192.168.100.253}, I was able to retrieve the following information about what services on the target machine are running on what ports:
\newline
\newline

\begin{adjustwidth}{-2cm}{}
\begin{tabular}{ |c|c|c|l| }
\hline
Port Number & State & Service & Version \\
\hline
21/tcp & open & ftp & ProFTPD 1.3.5 \\
\hline
22/tcp & open & ssh & OpenSSH 6.6.1p1 Ubuntu 2ubuntu2.10 (Ubuntu Linux; protocol 2.0) \\
\hline
80/tcp & open & http & Apache httpd 2.4.7 \\
\hline
111/tcp & open & rpcbind & 2-4 (RPC \# 100000) \\
\hline
139/tcp & open & netbios-ssn & Samba smbd 3.X - 4.X (workgroup: WORKGROUP) \\
\hline
445/tcp & open & netbios-ssn & Samba smbd 3.X - 4.X (workgroup: WORKGROUP) \\
\hline
3306/tcp & open & mysql & MySQL (unauthorized) \\
\hline
6667/tcp & open & irc & UnrealIRCd \\
\hline
6697/tcp & open & irc & UnrealIRCd \\
\hline
8067/tcp & open & irc & UnrealIRCd \\
\hline
8080/tcp & open & http & Jetty 8.1.7.v20120910 \\
\hline
8181/tcp & open & http & WEBrick httpd 1.3.1 (Ruby 2.3.6 (2017-12-14)) \\
\hline
49307/tcp & open & status & 1 (RPC \# 100024) \\
\hline
\end{tabular}
\end{adjustwidth}

\newpage

\section{Vulnerability Identification}
\paragraph{}
The following services on the target machine had exploitable vulnerabilities:

\subsection{ProFTPD 1.3.5}
\paragraph{}
Version 1.3.5 of ProFTPD allows attackers to read and write to arbitrary files via the \texttt{site cpfr} and \texttt{site cpto} commands.\footnote{link to cve here}
Metasploit has the module \texttt{exploit/unix/ftp/proftpd\_modcopy\_exec} which exploits this flaw.
By using this module and specifying the \texttt{SITEPATH} option to be \texttt{/var/www/html}, I was able to remotely log in to the machine as the \texttt{www-data} user.
While logged in, I was able to download files in the \texttt{/var/www/html} directory\footnote{Among those files was the administrator account details of the phpMyAdmin service} and traverse the filesystem after starting an interactive shell.

\newpage

\section{Information Extraction}
\paragraph{}
Lorem ipsum.

\newpage

\section{Security Recommendations}
\paragraph{}
Lorem ipsum.

\end{document}
