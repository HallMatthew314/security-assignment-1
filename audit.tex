\documentclass{article}
\usepackage{changepage}

\title{IN618 - Assignment 1}
\author{Matthew Hall}
\date{}

\begin{document}
\pagenumbering{gobble}
\maketitle
\newpage
\pagenumbering{arabic}

\tableofcontents
\newpage

\section{Host Discovery}
\paragraph{}
I used the following commands during the host discorvery process:

\texttt{nmap -sn 192.168.100.1}

Identify hosts with IP addresses in the network space of 192.168.100.0/24

\texttt{nmap -O 192.168.100.1}

Attempt to identify the operating system of the host with address 192.168.100.1

\texttt{nmap -O 192.168.100.115}

Attempt to identify the operating system of the host with address 192.168.100.115

\texttt{nmap -O 192.168.100.253}

Attempt to identify the operating system of the host with address 192.168.100.253

\newpage

\section{Information Gathering}
\paragraph{}
By use of the nmap command \texttt{nmap -sV 192.168.100.253}, I was able to retrieve the following information about what services on the target machine are running on what ports:
\newline
\newline

\begin{adjustwidth}{-2cm}{}
\begin{tabular}{ |c|c|c|l| }
\hline
Port Number & State & Service & Version \\
\hline
21/tcp & open & ftp & ProFTPD 1.3.5 \\
\hline
22/tcp & open & ssh & OpenSSH 6.6.1p1 Ubuntu 2ubuntu2.10 (Ubuntu Linux; protocol 2.0) \\
\hline
80/tcp & open & http & Apache httpd 2.4.7 \\
\hline
111/tcp & open & rpcbind & 2-4 (RPC \# 100000) \\
\hline
139/tcp & open & netbios-ssn & Samba smbd 3.X - 4.X (workgroup: WORKGROUP) \\
\hline
445/tcp & open & netbios-ssn & Samba smbd 3.X - 4.X (workgroup: WORKGROUP) \\
\hline
3306/tcp & open & mysql & MySQL (unauthorized) \\
\hline
6667/tcp & open & irc & UnrealIRCd \\
\hline
6697/tcp & open & irc & UnrealIRCd \\
\hline
8067/tcp & open & irc & UnrealIRCd \\
\hline
8080/tcp & open & http & Jetty 8.1.7.v20120910 \\
\hline
8181/tcp & open & http & WEBrick httpd 1.3.1 (Ruby 2.3.6 (2017-12-14)) \\
\hline
49307/tcp & open & status & 1 (RPC \# 100024) \\
\hline
\end{tabular}
\end{adjustwidth}

\newpage

\section{Vulnerability Identification}
\paragraph{}
The following services on the target machine had exploitable vulnerabilities:

% Port 21/tcp
\subsection{ProFTPD 1.3.5}
\paragraph{}
Version 1.3.5 of ProFTPD has vulnerability \emph{CVE-2015-3306}, which allows attackers to read and write to arbitrary files via the \texttt{site cpfr} and \texttt{site cpto} commands.\footnote{https://cve.mitre.org/cgi-bin/cvename.cgi?name=CVE-2015-3306}
Metasploit has the module \texttt{exploit/unix/ftp/proftpd\_modcopy\_exec} which exploits this flaw.
By using this module and specifying the \texttt{SITEPATH} option to be \texttt{/var/www/html}, I was able to remotely log in to the machine as the \texttt{www-data} user.
While logged in, I was able to download files in the \texttt{/var/www/html} directory\footnote{Among those files was the administrator account details of the phpMyAdmin service} and traverse the rest of the filesystem after starting an interactive shell.

% Ports 6667/tcp, 6697/tcp and 8067/tcp
\subsection{UnrealIRCd 3.2.8.1}
\paragraph{}
During the period of time from November 2009 to June 2010, version 3.2.8.1 of UnrealIRCd contained \emph{CVE-2010-2075}: a backdoor allowing remote attackers to login to the system and execute arbitrary commands.\footnote{https://cve.mitre.org/cgi-bin/cvename.cgi?name=CVE-2010-2075}
This vulnerability was only present in copies of the software downloaded from particular mirror sites.
This backdoor, if present, can be exploited with metasploit module \texttt{exploit/unix/irc/unreal\_irc\_3281\_backdoor}.
Using this exploit, I was able to remotely log in to the machine as the \texttt{boba\_fett} user\footnote{This user had a text file in their home directory containing the login information for the \texttt{darth\_vader} user.} and use the machine as if I were in an ssh session, all without having to supply a password.

\newpage

\section{Information Extraction}
\paragraph{}
As requested, I have located and extracted both the login details of every user account on the system and all information regarding the plans of the Galactic Empire.

\subsection{Login Details}

\begin{adjustwidth}{0cm}{}
\begin{tabular}{ |c|c|c|c| }
\hline
\textbf{Username} & \textbf{Password} & \textbf{Password Hash} & \textbf{Password Salt} \\
\hline
admiral\_ozzel & (unknown) & (unknown) & (unknown) \\
\hline
admiral\_piett & (unknown) & (unknown) & (unknown) \\
\hline
boba\_fett & (unknown) & (unknown) & (unknown) \\
\hline
captain\_needa & (unknown) & (unknown) & (unknown) \\
\hline
darth\_sidious & (unknown) & (unknown) & (unknown) \\
\hline
darth\_vader & daddy\_issues-7733 & (unknown) & (unknown) \\
\hline
death\_star\_admin & (unknown) & (unknown) & (unknown) \\
\hline
emperor\_palpatine & (unknown) & (unknown) & (unknown) \\
\hline
general\_veers & (unknown) & (unknown) & (unknown) \\
\hline
imperial\_guards & (unknown) & (unknown) & (unknown) \\
\hline
storm\_trooper\_1 & (unknown) & (unknown) & (unknown) \\
\hline
storm\_trooper\_2 & (unknown) & (unknown) & (unknown) \\
\hline
storm\_trooper\_3 & (unknown) & (unknown) & (unknown) \\
\hline
storm\_trooper\_4 & (unknown) & (unknown) & (unknown) \\
\hline
storm\_trooper\_5 & (unknown) & (unknown) & (unknown) \\
\hline
\end{tabular}
\end{adjustwidth}

\newpage

\section{Security Recommendations}
\paragraph{}
Lorem ipsum.

\end{document}
